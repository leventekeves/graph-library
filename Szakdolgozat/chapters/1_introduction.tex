\Chapter{Bevezetés}

Az adatok kezelése, mint a számítógépes alkalmazások egy gyakran megjelenő feladata, szükségessé tette, hogy  külön adatmodelleket definiáljanak, adatbázis kezelő rendszereket hozzanak létre.
Az egyik legnépszerűbb adatmodell a relációs modell. Egy alkalmazás elkészítése esetében kézenfekvő módon adódhat, hogy az adatok szervezése ilyen formában történjen.

A relációs modell mellett előfordulnak további modellek, paradigmák. A dolgozat azt mutatja be, hogy a gráfadatbázisok használatára webalkalmazások készítése esetében milyen formában, milyen előnyökkel és hátrányokkal van lehetőség.

% Ezt elegendő csak a végén összerakni, amikor már lehet látni egyben, hogy pontosan mi és hogy készült el (nem csak program, de dolgozat szintjén is).

% TODO: Itt a gráfadatbázisokról kellene úgy általában egy rövid (esetleg akár kissé promóció jellegű) leírásnak lennie.

% TODO: Röviden említeni kellene, hogy könyvtáras példáról lesz szó. Indokolni kellene, hogy a könyvtáras példa miért tünt megfelelőnek.

% TODO: Fel kellene sorolni, hogy milyen összehasonlításokra, és vizsgálatokra kerül majd sor.

A dolgozat négy nagyobb fejezetből áll. Ezek közül az első a \textit{Gráfadatbázisok} című fejezet.
Ebben a fejezetben először a különböző adatbázis paradigmák kerülnek bemutatásra. Ezt követően az adatok gráfokkal való leírásáról lesz szó, majd néhány elterjedt gráfadatbázis kerül bemutatásra

\bigskip

A 3. fejezetben a gráfadatbázisok vizsgálatához elkészített minta alkalmazás kerül részletes bemutatásra. Először az alkalmazás felhasználóinak szerepkörei, illetve a használati esetei kerülnek bemutatásra. Ezt követően az alkalmazás funkcióiról lesz szó, azaz hogy egy felhasználó mi mindent tud csinálni az alkalmazáson belül. Ez után az alkalmazás architektúrájának ismeretesére és az API bemutatására fog sor kerülni. Ezt követi majd az alkalmazás és a backend megvalósításának leírása.

\bigskip

A 4. fejezet a Neo4j gráfadatbázis bemutatásával foglalkozik. Először a Neo4j által nyújtott szolgáltatások kerülnek ismertetésre, majd egy rövid leírást olvashatunk arról, hogy hogyan lehet adatokat tárolni a Neo4j féle gráfadatbázisban. Ez után a Neo4j által használt Cypher lekérdező nyelv kerül összehasonlításra az SQL nyelvvel. A fejezet az alkalmazás és az adatbázis összekapcsolásának menetéről, valamint a magasabb szintű hozzáférési módok ismertetéséről szól.

\bigskip

Az 5. fejezet az \textit{Optimalizálás és adatbázis menedzsment} címet viseli. Ebben a fejezeten webalkalmazások és adatbázisok jogosultságkezelése és gyorsítótárazása kerül röviden ismertetésre, majd a Neo4j gráfadatbázis kongifurálási lehetőségeit mutatja be a dolgozat. Végül a Neo4j által biztosított biztonsági mentési lehetőségekről olvashatunk.
