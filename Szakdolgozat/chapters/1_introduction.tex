\Chapter{Bevezetés}

% Ezt elegendő csak a végén összerakni, amikor már lehet látni egyben, hogy pontosan mi és hogy készült el (nem csak program, de dolgozat szintjén is).

% TODO: Itt a gráfadatbázisokról kellene úgy általában egy rövid (esetleg akár kissé promóció jellegű) leírásnak lennie.

% TODO: Röviden említeni kellene, hogy könyvtáras példáról lesz szó. Indokolni kellene, hogy a könyvtáras példa miért tünt megfelelőnek.

% TODO: Fel kellene sorolni, hogy milyen összehasonlításokra, és vizsgálatokra kerül majd sor.

A dolgozat négy nagyobb fejezetből áll. Ezek közül az első a Gráfadatbázisok című fejezet.
Ebben a fejezetben először a különböző adatbázis paradigmák kerülnek bemutatásra. Ezt követően az adatok gráfokkal való leírásáról lesz szó, majd néhány elterjedt gráfadatbázis kerül bemutatásra

\bigskip

A minta alkalmazás fejezet során az elkészült alkalmazás kerül részletes bemutatásra. Először az alkalmazás szerepkörei, illetve használati esetei kerülnek bemutatásra. Ezt követően az alkalmazás funkcióiról lesz szó, azaz hogy egy felhasználó mi mindent tud csinálni az alkalmazáson belül. Ez után az alkalmazás architektúrájának ismeretesére és az API bemutatására fog sor kerülni. Ezt követi majd az alkalmazás és a backend megvalósításának leírása.

\bigskip

Ezt követi a Neo4j gráfadatbázist bemutató fejezet. Először a Neo4j által nyújtott szolgáltatások lesznek ismertetve, majd egy rövid leírás arról, hogy hogyan lehet adatokat tárolni a Neo4j féle gráfadatbázisban. Ezt után a Neo4j által használt Cypher lekérdező nyelv kerül összehasonlításra az SQL-el. A fejezet az alkalmazás és az adatbázis összekapcsolásának menetéről, valamint a magasabb szintű hozzáférési módok ismertetéséről fog szólni.

\bigskip

Az utolsó nagyobb fejezet Optimalizálás és adatbázis menedzsment névre hallgat. Ebben a fejezeten webalkalmazások és adatbázisok jogosultságkezelése és gyorsítótárazása lesz röviden ismertetve, majd a Neo4j gráfadatbázis kongifurálási lehetőségei kerülnek bemutatásra. Végül pedig a Neo4j által biztosított biztonsági mentési lehetőségek lesznek bemutatva.