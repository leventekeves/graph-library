\Chapter{Gráfadatbázisok}

% TODO: Itt amit csak lehet hivatkozni kellene. A dolgozatban lévő hivatkozások jelentős része kerülhet ebbe a fejezetbe.
Ebben a fejezetben szó lesz arról, hogy milyen adatbázis paradigmák vannak, hogyan lehet adatokat gráfokkal leírni,  valamint hogy milyen elterjedt gráfadatbázisok vannak.

\Section{Paradigmák}

% TODO: Itt ki kellene fejteni, hogy milyen jellemző adatbázis paradigmák vannak. (Múltkor a 4 fő típust említettem, de Lehet itt 7 is.)

% https://tudip.com/blog-post/7-database-paradigms/

% https://arato.inf.unideb.hu/ispany.marton/Database/Lectures2020/noSQL.pdf
% http://eta.bibl.u-szeged.hu/5031/7/EFOP343_A6_7e_BigData-nosql-over-hadoop-SPOC.pdf
% https://inf.mit.bme.hu/sites/default/files/publications/nosql-konyvfejezet.pdf
Ebben az alfejezetben 7 adatbázis paradigma kerül bemutatásra. \cite{paradigms} \cite{nosql1} \cite{nosql2} \cite{nosql3}

\subsection{Kulcs-érték adatbázis}

Ez az adatbázis típus kulcs-érték párokban tárolja az adatokat.  Az adatokat a kulcsok segítségével gyorsan el lehet érni, viszont ezen kívül jellemzően nem lehet másfajta lekérdezéseket végrehajtani. Ilyen adatbázisok például a Redis, a Memcached és a Riak.

\subsection{Oszlopalapú adatbázis}
Szokás még oszlopcsalád adatbázisnak, vagy oszlop adatbázisnak is nevezni. Az oszlopalapú adatbázis hasonlít a kulcs-érték adatbázishoz, ugyanis itt is kulcs-érték páronként vannak tárolva az adatok, de az érték oszlopokban kerülnek tárolásra. Egy oszlophoz jellemzően tartozik egy kulcs, egy érték, és egy időbélyeg. Nincs hozzá séma, ezért csak szervezetlen adatok tárolására alkalmas. A CQL (Cassandra Query Language) segítségével lehet lekérdezéseket írni, viszont nincsenek join műveletek. Ilyen adatbázisok például a Cassandra, az HBase és a BigTable.

\subsection{Dokumentum adatbázis}
A dokumentum adatbázis dokumentumok tárolására és lekérésére szolgálnak. A dokumentumok önleíróak, hierarchikus szerkezetűek, kollekciókat és skalár értékeket tartalmazhatnak. A dokumentumok lehetnek például JSON, XML vagy YAML formátumokban. Ilyen adatbázisok például a MongoDB, a CouchDB és a Firestore.

\subsection{Relációs adatbázis}
% Bővebben?
A relációs adatbázis a relációs adatmodellen alapszik. Ilyen adatbázisok például az Oracle Database, a MySQL és a Microsoft SQL Server.

\subsection{Gráf adatbázis}
A gráf adatbázis gráfok hatékony tárolására és azokon végzett műveletek gyors végrehajtására ad lehetőséget. Az adatokat csomópontok és élek formájában tárolja. Lehetőséget biztosít komplex lekérdezésekre. Ilyen adatbázisok például a Neo4j, az Amazon Neptune és az Azure Cosmos DB.

\subsection{Teljes szövegű kereső motor}
Teljes szövegű kereső motorra épülő adatbázisban a dokumentumok tartalmát indexelik. Az adatbázisban való kereséskor a keresés az indexen keresztül fog zajlani. Ilyen keresőmotorok például az Apache Lucene és az Elasticsearch

\subsection{Több modelles adatbázis}
Több modelles adatbázis lehetőséget biztosít több fajta adatmodell együttes használatára. Ilyen adatbázisok például a Fauna és az ArangoDB.

% https://neo4j.com/docs/getting-started/current/graphdb-concepts/#graphdb-concepts
\Section{Adatok leírása gráfokkal}
% TODO: Ki kellene térni arra, hogy milyen alternatívák vannak az adattárolásra. Például, hogy
% - Mi kerülhet a csomópontba és az élekbe. (Tehát, hogy melyik változatnál éppen hol van az adata.)
% - Ki kellene térni a típusosság problémakörére.

Gráfadatbázisok esetében az adatok csomópontok, és az azokat összekötő kapcsolatok formájában kerülnek tárolásra. \cite{adatok-leirasa} A csomópontokhoz és a kapcsolatokhoz tartozhatnak tulajdonságok is. Ezek kulcs-érték párok, amik az adott csomóponthoz vagy kapcsolathoz tartozó további információkat tárolnak. A csomópontokat el lehet látni címkékkel is, amik csoportosítják a csomópontokat. Általában a csomópont egy főnév, a kapcsolat egy ige.

\bigskip

Egy csomópont tulajdonságát bizonyos esetekben célszerű lehet külön csomópontként reprezentálni. Például egy gépjárműveket tartalmazó adatbázisban a gépjárművek márkáját lehetne tárolni tulajdonságként a gépjármű csomópontban, de célszerű lehet egy külön csomópontot létrehozni a márkának, köztük egy kapcsolattal.

\Section{Elterjedt gráfadatbázisok}

% TODO: Itt kellene összegyűjteni 6-8 elterjedt, vagy valamilyen szempontból jelentős gráfadatbázist.
Számos elérhető gráfadatbázis szolgáltatás létezik. Ebben az alfejezetben röviden bemutatásra kerül közülük néhány.

% https://neo4j.com/docs/getting-started/current/
\subsection{Neo4j}
A Neo4j a Neo4j, Inc. által fejlesztett gráfadatbázis. \cite{neo4j} Az architektúrát úgy tervezték, hogy az optimális legyen a csomópontok és a kapcsolatok kezelésére, tárolására és bejárására. Tulajdonsággráf megközelítést alkalmaz, ami előnyös bejárási teljesítmény és műveleti futásidő szempontjából. A Cypher lekérdező nyelvet használja, ami SQL jellegű.

% https://docs.aws.amazon.com/neptune/latest/userguide/intro.html
\subsection{Amazon Neptune}
Az Amazon Neptune az Amazon.com, Inc. által fejlesztett gráfadatbázis. \cite{neptune} A Neptune egy gyors, megbízható és teljesen felügyelt gráfadatbázis, amely megkönnyíti a szorosan összekapcsolt adatkészletekkel működő alkalmazások létrehozását és futtatását. A Neptune magja egy erre a célra épített, nagy teljesítményű gráf adatbázis motor. Ez a motor több milliárd kapcsolatok tárolására és a grafikon ezredmásodperces késleltetéssel történő lekérdezésére van optimalizálva. A Neptune támogatja az Apache TinkerPop Gremlin és a W3C SPARQL gráflekérdező nyelveit.

% https://docs.microsoft.com/en-us/azure/cosmos-db/introduction
\subsection{Azure Cosmos DB}
Az Azure Cosmos DB a Microsoft Corporation által fejlesztett gráfadatbázis. \cite{azure} Nagyon gyors válaszidővel, és automatikus és azonnali skálázhatósággal van felszerelve. 99,99\%-os SLA-t (Service-level agreement/Szolgáltatási Szint Megállapodás) kínál. Használható hozzá az Apache TinkerPop Gremlin lekérdező nyelv. 

% https://docs.tigergraph.com/tigergraph-server/current/intro/
\subsection{Tigergraph}
A TigerGraph a TigerGraph által fejlesztett gráfadatbázis. \cite{tigergraph} A TigerGraph biztosít gyors adatbetöltési sebességet grafikonok készítéséhez, párhuzamos gráf-algoritmusok gyors végrehajtását, valós idejű frissítéseket és beillesztéseket a REST segítségével, valamint képes a valós idejű elemzések egyesítésére nagyszabású offline adatfeldolgozással. A GSQL szoftvert használja az adatbázis menedzselésére.

% https://cambridgesemantics.com/anzograph/
% https://docs.cambridgesemantics.com/anzograph/v2.5/userdoc/features.htm
\subsection{AnzoGraph DB}
AnzoGraph DB a Cambridge Semantics által fejlesztett gráfadatbázis. Az AnzoGraph egy natív, masszívan párhuzamos feldolgozású gráf  OLAP (Online Analytics Processing) adatbázis. \cite{anzograph1} \cite{anzograph2} Az adatok natív gráf formátumban kerülnek tárolásra a lemezen vagy a memóriában. A gráf  OLAP technológia lehetővé teszi a felhasználók számára a grafikonok adatainak interaktív megtekintését, elemzését és frissítését. Használhatóak hozzá a SPARQL és a Cypher lekérdező nyelvek.

% https://dgraph.io/docs/dgraph-overview/
\subsection{Dgraph}
A Dgraph a Dgraph Labs által fejlesztett gráfadatbázis. \cite{dgraph} A Dgraph egy horizontálisan méretezhető é GraphQL. A Dgraph a modern alkalmazások és webhelyek működéséhez szükséges nagy tranzakciós terhelésre készült. A GraphQL lekérdező nyelvre épülő DQL-t (Dgraph Query Language) használja.