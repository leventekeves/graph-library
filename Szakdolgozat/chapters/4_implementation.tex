\Chapter{Adatkezelés Neo4j segítségével}
% TODO: Egy rövid felvezetés kellene, hogy a Neo4j-s megoldások elsődlegesen a relációs adatmodellel kerülnek összehasonlításra.

A mintaalkalmazás adatainak tárolása egy Neo4j által biztosított gráfadatbázisban történik. Ebben a fejezetben szó lesz arról, hogy milyen szolgáltatásokat biztosít a Neo4j a felhasználók számára, hogyan írhatjuk le az adatokat, hogy valósulnak meg a lekérdezések a Cypher lekérdező nyelv segítségével, hogy hogyan lehet kapcsolatot létesíteni az adatbázis és a mintaalkalmazás között, valamint arról, hogy a Neo4j milyen magasabb rendű hozzáférési módot biztosít.

\Section{Szoftveres eszközök}
% TODO: Itt röviden be kellene mutatni, hogy milyen szoftveres eszközök fordulnak elő. Értem ez alatt magát a db szerver implementációt, a különféle klienseket, adminisztrációs eszközöket.

A Neo4j számos szoftveres eszközt biztosít a gráfokkal való adatkezelés lebonyolításához. Ebben az alfejezetben ezen eszközök rövid bemutatására fog sor kerülni.

% https://go.neo4j.com/rs/710-RRC-335/images/Neo4j-product-brief-database-US-EN.pdf
\subsection{Neo4j Graph Database}

A \textit{Neo4j Graph Database} a Neo4j Platform központi része. A Neo4j széles körben elterjedt, sikerrel alkalmazzák különféle tudományos, ipari és vállalati területeken, beleértve az élettudományokat, a közműveket, a pénzügyi szolgáltatásokat, a kiberbiztonságot és még sok mást \cite{neo4j-graph-database}.

A Neo4j Graph Database legfontosabb tulajdonságai közé tartozik
\begin{itemize}
      \item a gráfok automatikus, dinamikus átméretezése,
      \item a kiváló teljesítmény,
      \item a megbízható működés az üzemeltetés során,
      \item a felhő alapú rendszerekhez illeszkedő kialakítása, 
      \item a hatékony fejlesztés segítése.
\end{itemize}
A következő szakaszokban az eszközkészletének néhány eleme kerül említésre.

\subsection{Neo4j AuraDB}

A \textit{Neo4j AuraDB} egy felhőalapú gráfadatbázis szolgáltatás. 

% https://neo4j.com/developer/graph-data-science/
\subsection{Neo4j Graph Data Science}

A \textit{Graph Data Science} segítségével ismeretekre lehet szert tenni az adatok összefüggéseiből és struktúráiból, jellemzően előrejelzésekhez . Olyan eszköztárat taralmaz, amely segít az adatkutatóknak megválaszolni kérdéseket és megmagyarázni az eredményeket grafikonokon ábrázolt adatok segítségével \cite{neo4j-graph-data-science}.

\subsection{Neo4j Developer Tools}

A Neo4j két fejlesztői felületet biztosít: egy letölthető és telepíthető asztali alkalmazást, és egy böngészőből futtatható alkalmazát.

\subsubsection{Neo4j Desktop}

A \textit{Neo4j Desktop} egy fejlesztői IDE (\textit{Integrated Development Environment}) vagy felügyeleti környezet \cite{neo4j-desktop}.
Tetszőleges számú lokális projektet és adatbázist lehet vele kezelni, valamint lehet vele csatlakozni távoli Neo4j szerverekhez is. 

A Neo4j Desktop által kezelt adatbázisok konfigurálhatók, frissíthetők és karbantarthatók a felhasználói felületen keresztül, parancssor nélkül. Lehetőséget biztosít Neo4j bővítmények telepítésére.

\subsubsection{Neo4j Browser}

A \textit{Neo4j Browser} egy böngészőből futtatható Neo4j interface lekérdezésekhez és adatok megjelenítéséhez. Cypher-szerkesztőt kínál szintaktikai kiemeléssel, kódkiegészítéssel és figyelmeztetésekkel, amelyek segítenek a Cypher lekérdezések írásakor \cite{neo4j-browser}.

% https://neo4j.com/developer/neo4j-bloom/
% https://neo4j.com/docs/bloom-user-guide/current/
% https://neo4j.com/docs/bloom-user-guide/current/about-bloom/ itt van még pár dolog róla
\subsection{Neo4j Bloom}

A \textit{Neo4j Bloom} egy olyan eszköz, amely megjeleníthetővé teszi az adatokat egy gráfban, valamint lehetőséget biztosít lekérdezésekre lekérdező nyelv vagy programozási nyelv ismerete nélkül. A Bloom igénybe vehető a Neo4j Desktop és Neo4j Browser szolgáltatásokon keresztül \cite{neo4j-bloom1}, \cite{neo4j-bloom2}.

% https://neo4j.com/developer/graphql/
% https://neo4j.com/docs/graphql-manual/current/introduction/
\subsection{Neo4j GraphQL Library}

A \textit{Neo4j GraphQL Library} egy rendkívül rugalmas, kevés kódolást igénylő (\textit{low code}), nyílt forráskódú JavaScript könyvtár, amely lehetővé teszi a gyors API-fejlesztést.

A Neo4j gráf adatbázissal a GraphQL Library egyszerűvé teszi az alkalmazások számára, hogy az alkalmazás adatokat natívan gráfként kezeljék a frontend-től egészen a tárolásig, elkerülve ezzel a séma duplikációt, és probléma mentes integrációt biztosítva a frontend és a backend között.

A könyvtár TypeScripten készült. A "Séma az első" (\textit{schema first}) paradigma lehetővé teszi a fejlesztők számára, hogy a szükséges alkalmazásadatokra összpontosítsanak, miközben a könyvtár gondoskodik az API felépítésének nehezéről  \cite{neo4j-graphql-library1}, \cite{neo4j-graphql-library2}.

\Section{Az adatok leírása}

% TODO: Fel kell sorolni, hogy milyen adatokat hogy lehet tárolni a gráfban.
% https://neo4j.com/docs/getting-started/current/graphdb-concepts/#graphdb-concepts - not cited
% https://neo4j.com/docs/cypher-manual/current/syntax/values/

A Neo4j egy típusos adatbázis, ami az alábbi típusokat tudja kezelni \cite{neo4j-values}:
\begin{itemize}
    \item Tulajdonság típusok: Integer, Float, String, Boolean, Point, Date, Time, LocalTime, DateTime, LocalDateTime, Duration,
    \item Strukturális típusok: Node, Relationship, Path,
    \item Összetett típusok: List, Map.
\end{itemize}

Neo4j-ben a csomópontok egyedeket reprezentálnak \cite{adatok-leirasa}. A csomópontokhoz tartozhatnak úgynevezett címkék, amik leírják, hogy milyen típusú az adott csomópont. Egy csomóponthoz tartozhat nulla, egy, vagy több címke. Például a mintaalkalmazásban található könyvek \textit{Book} címkével ellátott csomópontokként kerülnek ábrázolásra az adatbázisban.

Az élek két csomópont közötti kapcsolatot reprezentálnak. Egy kapcsolatnak mindig van iránya és típusa. Például a mintaalkalmazásban egy könyv felhasználó általi értékelését követően a felhasználó és a könyv csomópontok között létrejön egy \textit{Rated} kapcsolat.

Csomópontok és kapcsolatok rendelkezhetnek tulajdonságokkal, amik kulcs-érték párok. Például egy könyvhöz tartozhat szerző tulajdonság.

\Section{Lekérdezések}

% TODO: Sorban végig kellene nézni, hogy a különbző lekérdezések hogyan valósíthatók meg. Egymás után/mellett szép lehet megmutatni azt, hogy egy lekérdezés hogyan néz ki SQL-ben és CQL-ben. (Itt azt át kell gondolni, hogy mi legyen a szisztéma, vagyis hogy módszeresen melyik nyelv legyen először.)

% https://neo4j.com/developer/cypher/

Neo4j-ben a lekérdezések a \textit{Cypher Query Language} segítségével valósulnak meg. A Cypher egy SQL által inspirált lekérdező nyelv, ezért ebben az alfejezetben egyaránt bemutatásra kerül, hogy bizonyos műveletek hogyan valósíthatók meg Cypherrel és SQL-el \cite{cypher}, \cite{sql}.

% https://neo4j.com/developer/cypher/guide-sql-to-cypher/

% CQL               & SQL

\subsection{Kiválasztás}
% match             & select
% optional match    & outer join
% return            & -
Cypher-ben az adatok kiválasztása a \texttt{MATCH} paranccsal történik, amiket a kiválasztást követően a \texttt{RETURN} paranccsal lehet megkapni. A következő parancs lekérdezi az adatbázisban található összes könyvet.
\begin{java}
MATCH (n:Book) RETURN n
\end{java}
Az alábbi parancs pedig a Stephen King által írt könyveket fogja lekérdezni.
\begin{java}
MATCH (n:Book{author: "Stephen King"}) RETURN n
\end{java}
Az \texttt{OPTIONAL MATCH} parancs hasonlóan működik mint a \texttt{MATCH}, annyi eltéréssel, hogy ha nincs találat, akkor az \texttt{OPTIONAL MATCH} null értéket fog használni a minta hiányzó részeinél. A következő parancs az adatbázisban található könyveket fogja visszaadni, és ha van hozzájuk értékelés, akkor azoknak az átlagát is. Amennyiben nincs, akkor az átlag helyén null érték fog szerepelni.
\begin{java}
MATCH (b:Book) 
OPTIONAL MATCH (a)-[r:Rated]->(b) 
RETURN b.title, avg(r.rating)
\end{java}
SQL-ben a \texttt{MATCH} megfelelője a \texttt{SELECT} paranccsal, míg az \texttt{OPTIONAL MATCH} megfelelője az \texttt{OUTER JOIN}:
\begin{java}
SELECT * FROM Books;
SELECT * FROM Books WHERE Author='Stephen King';
SELECT Books.Title, AVG(Rated.Rating) FROM Books 
FULL OUTER JOIN Rated USING(BookID) GROUP BY Books.Title;
\end{java}

\subsection{Korlátozás}
% where             & where
A \texttt{WHERE} paranccsal korlátozásokat lehet hozzárendelni a \texttt{MATCH} és \texttt{OPTIONAL MATCH} parancsokhoz. A \texttt{WHERE} paranccsal lehet vizsgálni például egy csomópont vagy kapcsolat ID-jét, címkéjét és tulajdonságait, számok közötti egyenlőséget illetve egyenlőtlenséget, szöveges egyezést. Az alábbi parancs a 24-es ID-vel rendelkező könyv adatait fogja visszaadni.
\begin{java}
MATCH (b:Book) 
WHERE ID(b)=24
RETURN b
\end{java}

% \noindent SQL-ben

\subsection{Létrehozás}
% create            & insert into
% https://neo4j.com/docs/cypher-manual/current/clauses/create/
Cypher-rel az adatbázisban csomópontot és kapcsolatot a \texttt{CREATE} paranccsal lehet létrehozni. A következő lekérdezés létrehoz egy könyvet az adatbázisban a megadott tulajdonságokkal. A parancsban az \textit{n} egy változó, a \textit{Book} egy címke, míg az \textit{author} és a \textit{title} tulajdonságok. 
\begin{java}
CREATE(n:Book{
  author:"J. R. R. Tolkien",
  title:"The Two Towers"
})
\end{java}
Az alábbi parancs egy kapcsolatot fog létrehozni a megadott felhasználó és megadott könyv között a megadott névvel.
\begin{java}
MATCH (a:User), (b:Book) 
WHERE ID(a)=112 AND ID(b)=23
CREATE (a)-[r:Bookmarked]->(b) 
\end{java}
SQL-ben adat rögzítéséhez először egy táblát kell létrehozni, majd ezt követően lehet azt feltölteni adatokkal.
\begin{java}
CREATE TABLE Books(
  BookID INT PRIMARY KEY NOT NULL,
  Author VARCHAR(30),
  Title VARCHAR(30)
);
INSERT INTO Books
VALUES (1, 'J. R. R. Tolkien', 'The Two Towers');
\end{java}
A Cypher-beli kapcsolatot SQL-ben egyedi és idegen kulcsokkal lehet megvalósítani. Az alábbi példában kapcsolótábla kerül alkalmazásra:
\begin{java}
CREATE TABLE Bookmarked(
  UserID INT, 
  FOREIGN KEY(UserID) REFERENCES Users(UserID),
  BookID INT, 
  FOREIGN KEY(BookID) REFERENCES Books(BookID)
);
INSERT INTO Bookmarked VALUES (112, 23);
\end{java}

\subsection{Törlés}
% delete            & delete
Csomópontok és kapcsolatok törlésére a \textbf{DELETE} paranccsal van lehetőségünk. Az alábbi lekérdezés törölni fogja a megadott csomópontot, amennyiben az nem rendelkezik kapcsolatokkal.
\begin{java}
MATCH (n:Book) 
WHERE ID(n)=24 
DELETE n
\end{java}
Amennyiben olyan csomópontot szeretnénk törölni, amihez tartozik kapcsolat, akkor a \texttt{DETACH DELETE} parancsot kell használni.
\begin{java}
MATCH (n:Book) 
WHERE ID(n)=24
DETACH DELETE n
\end{java}
A következő parancs megszünteti a \textit{Bookmarked} kapcsolatot a megadott felhasználó és a megadott könyv között.
\begin{java}
MATCH (a:User)-[r:Bookmarked]->(b:Book) 
WHERE ID(a)=12 AND ID(b)=14
DELETE r
\end{java}
SQL-ben a törlés a \texttt{DELETE} paranccsal történik.
\begin{java}
DELETE FROM Books 
WHERE BookID = 24;
\end{java}

\subsection{Rendezés}
% order by          &  order by
Cypher-ben az \texttt{ORDER BY} paranccsal lehet a találatokat rendezni. Az alábbi parancs a könyv darabszáma alapján csökkenő sorrendben fogja visszaadni a találatokat.
\begin{java}
MATCH (n:Book) 
RETURN n.stock 
ORDER BY n.stock DESC
\end{java}
SQL-ben szintén az \texttt{ORDER BY} paranccsal történik a rendezés.
\begin{java}
SELECT * 
FROM Books 
ORDER BY Stock DESC;
\end{java}

\subsection{Kihagyás}
% skip              & skip
Cypher-ben a \texttt{SKIP} paranccsal át lehet ugrani a megadott mennyiségű első találatot. A következő parancs a könyv darabszáma alapján fogja rendezni a találatokat csökkenő sorrendben, majd ezek közül kihagy 1-et, és visszaadja a többit.
\begin{java}
MATCH (n:Book) 
RETURN n.stock 
ORDER BY n.stock DESC
SKIP 1
\end{java}
SQL-ben elem kihagyására az \texttt{OFFSET} paranccsal van lehetőség.
\begin{java}
SELECT * 
FROM Books 
ORDER BY Stock DESC
OFFSET 1;
\end{java}

\subsection{Limitálás}
% limit             & offset
Cypher-ben a \texttt{LIMIT} paranccsal meg lehet adni, hogy hány találatot szeretnénk megkapni. Az alábbi parancs a könyv darabszáma alapján fogja rendezni a találatokat csökkenő sorrendbe, de csak az első két találatot adja vissza.
\begin{java}
MATCH (n:Book) 
RETURN n.stock 
ORDER BY n.stock DESC
LIMIT 2
\end{java}
SQL-ben hasonlóképpen használható a \texttt{LIMIT} parancs.
\begin{java}
SELECT *
FROM Books 
ORDER BY Stock DESC
LIMIT 2;
\end{java}
Alternatívaként használható az alábbi, \texttt{FETCH} paranccsal megadott lekérdezés is.
\begin{java}
SELECT *
FROM Books 
ORDER BY Stock DESC;
FETCH FIRST 2 ROWS ONLY;
\end{java}

A \texttt{SKIP} és \texttt{LIMIT}, illetve \texttt{OFFSET} és \texttt{FETCH} parancsokat gyakran együtt használják a webalkalmazásokban a lapozás (\textit{pagination}) megvalósításához.

\subsection{Frissítés}
% set               & update/set
Cypher-ben a \texttt{SET} paranccsal lehet tulajdonságokat és címkéket létrehozni, értéküket megváltoztatni, illetve törölni. Az alábbi parancs a megadott ID-jű könyv kategóriáját \textit{Fantasy}-ra fogja beállítani.
\begin{java}
MATCH (n:Book) 
WHERE ID(n)=24
SET n.category="Fantasy"
\end{java}
SQL-ben ez az \texttt{UPDATE} paranccsal történik.
\begin{java}
UPDATE Books
SET Category = 'Fantasy'
WHERE BookID = 24;
\end{java}

% \subsubsection{Egyebek}
% merge             & merge?
% remove

\Section{Kapcsolódás az adatbázishoz}
% TODO: Ezt elég csak röviden részletezni.

A Neo4j-hez igénybe vehető számos hivatalos meghajtóprogram (\textit{driver}), amik egyszerűsítik egy alkalmazás és az adatbázis összekapcsolását. Az alábbi programozási nyelvekhez érhető el hivatalos kiadás:
\begin{itemize}
    \item .NET,
    \item Go,
    \item Java,
    \item JavaScript,
    \item Python.
\end{itemize}
JavaScript esetében a kapcsolódás az alábbi módon történik:
\begin{itemize}
\item Neo4j driver telepítése:
\begin{java}
npm install neo4j-driver
\end{java}
\item Neo4j driver importálása: 
\begin{java}
const neo4j = require("neo4j-driver");
\end{java}
\item Neo4j driver inicializálása:
\begin{java}
const driver = neo4j.driver(
  uri,
  neo4j.auth.basic(user, password)
);
\end{java}
\end{itemize}

\Section{Lekérdezések kiadása}
% TODO: Tulajdonképpen a Neo4j API-t kell röviden bemutatni.
% https://neo4j.com/docs/javascript-manual/current/session-api/
A mintaalkalmazásban a lekérdezések kiadására a \textit{Neo4j Driver}-ben található \textit{Session API} kerül felhasználásra \cite{neo4j-session-api}. Ahhoz, hogy elkezdhessük használni a \textit{Session API}-t, először is deklarálnunk kell a \texttt{session} objektumot:
\begin{java}
const session = driver.session()
\end{java}

Ezt követően a \texttt{run} függvény segítségével futtathatjuk a lekérdezéseket az alábbi módon:
\begin{java}
session.run(query, queryParams);
\end{java}
 
A \texttt{query} az maga a lekérdezés sztring formátumban, a \texttt{queryParams} pedig egy opcionális objektum, ami tartalmazza a lekérdezéshez tartozó paramétereket. Például, az alábbi példában azt a felhasználót keressük, akinek az ID-je megegyezik a \texttt{userId} változóban szereplő értékkel. A lekérdezésben a \texttt{\$userIdParam} jelzi, hogy oda egy paramétert vár.
\begin{java}
const query = `
  MATCH(n:User) 
  WHERE ID(n)=$userIdParam 
  RETURN n`;
  
const queryParams = {
  userIdParam: userId
};
\end{java}

A \texttt{run} függvény futtatása után a \texttt{then} függvénnyel lehet a kapott választ feldolgozni. Az alábbi példában ez kerül kiírásra:
\begin{java}
session.run(query, queryParams)
  .then(function (result) {
    console.log(result)
  }
);
\end{java}

% https://neo4j.com/docs/ogm-manual/current/introduction/
\Section{Magasabb szintű hozzáférési módok}
% TODO: Ide kerülhetne akár csak irodalomkutatás jelleggel, hogy milyen ORM jellegű megoldások fordulhatnak elő.
Magasabb hozzáférési módként a Neo4j objektum gráf leképzést (\textit{OGM, Object Graph Mapper}) ajánl. Az OGM a gráf csomópontjait és kapcsolatait képezi le objektumokra és hivatkozásokra az alábbi módon:
\begin{itemize}
    \item Az objektumpéldányokat csomópontokra képzi.
    \item Az objektumhivatkozásokat kapcsolatokra képzi.
    \item A JVM primitíveket csomópont- vagy kapcsolattulajdonságokra képzi.
\end{itemize}

Az OGM elvonatkoztatja az adatbázist, és kényelmes módot biztosít a tartománymodell grafikonon való megtartására és lekérdezésére alacsony szintű illesztőprogramok használata nélkül. Rugalmasságot biztosít a fejlesztő számára egyéni lekérdezésekhez, ha a Neo4j-OGM által generált lekérdezések nem elegendőek.

A Neo4j a \textit{Neo4j-OGM} könyvtárat biztosítja az OGM-hez. Ez egy Java könyvtár, amely a Neo4j használatával képes megőrizni a tartományobjektumokat. Cypher utasításokat használ a műveletek kezelésére.
