\pagestyle{empty}

\noindent \textbf{\Large CD Használati útmutató}

\vskip 1cm

\noindent A CD tartalmazza az alábbiakat:
\begin{itemize}
    \item a szakdolgozat PDF formátumban,
    \item a szakdolgozat \LaTeX\ forráskódja,
    \item a mintaalkalmazás forráskódja,
    \item tesztadatbázis,
    \item API dokumentáció,
    \item CD Használati útmutató.
\end{itemize}

\bigskip

A CD-n található mintaalkalmazás futtatásához a klienshez és a backendhez is telepíteni kell a szükséges csomagokat. Az alkalmazás gyökérkönyvtárából kiindulva:
\begin{itemize}
    \item Szerver: \texttt{npm install}
    \item Kliens: \texttt{cd client}, ezt követően \texttt{npm install}
\end{itemize}

Ez után külön terminálokon futtatni kell mindkettőt. Az alkalmazás gyökérkönyvtárából kiindulva:
\begin{itemize}
    \item Szerver: \texttt{npm start}
    \item Kliens: \texttt{cd client}, ezt követően \texttt{npm start}
\end{itemize}

A tesztadatbázis a \texttt{tesztadatbazis.dump} fájlban található. Ezt a fájlt egy Neo4j gráfadatbázisba kell importálni. Neo4j Aura esetében az adatbázis létrehozásakor kapott felhasználónevet, jelszót és URI-t az alkalamzás \texttt{server.js} fájlban kell megadni.

\bigskip

A könyvek borítóképei a \textit{FireBase Storage}-be kerülnek feltöltésre.  Mivel a borítóképek elérési útvonala függ a projekt nevétől, a tesztadatbázisban található könyvek borítóképei nem fognak megjelenni, ha a \textit{FireBase Storage}, amin a képek vannak, nem elérhető. Amennyiben az eredeti \textit{FireBase Storage} még aktív, akkor nincs további teendő. Ha inaktív, akkor ahhoz, hogy a borítóképek működjenek, egy új \textit{FireBase Storage}-t kell létrehozni, majd összekapcsolni vele az alkalmazást. Először létre kell hozni egy új \textit{FireBase} projektet, majd telepíteni kell a \texttt{firebase} csomagot. Ezt követően be kell jelentkezni, inicializálni a projektet, majd beállítani a \texttt{storage}-t. Az \texttt{AdminNewBook.js} fájlban kell átírni a \texttt{projectId}-t az aktuális projektnek megfelelően. Ezt követően az új könyvek már az új \textit{FireBase Storage}-be kerülnek feltöltésre. A régi könyvek borítóképei megtalálhatók a tesztadatbázis mappájában. Ahhoz, hogy ezek megjelenjenek, fel kell őket tölteni manuálisan, majd a könyvek borítóképeinek a linkjét át kell írni az adatbázisban az újakra.