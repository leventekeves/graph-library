\pagestyle{empty}

\noindent \textbf{\Large CD Használati útmutató}

\vskip 1cm

\noindent A CD tartalmazza az alábbiakat:
\begin{itemize}
    \item A szakdolgozat
    \item A szakdolgozat forráskódja
    \item A mintaalkalmazás forráskódja
    \item Tesztadatbázis
    \item API dokumentáció
    \item CD Használati útmutató
\end{itemize}

\bigskip

A CD-n található mintaalkalmazás futtatásához a klienshez és a backendhez is telepíteni kell a szükséges csomagokat. Az alkalmazás gyökérkönyvtárából kiindulva:
\begin{itemize}
    \item Szerver: npm install
    \item Kliens: cd client, ezt követően npm install
\end{itemize}

Ezután külön terminálokon futtatni kell mindkettőt. Az alkalmazás gyökérkönyvtárából kiindulva:
\begin{itemize}
    \item Szerver: npm start
    \item Kliens: cd client, ezt követően npm start
\end{itemize}

A tesztadatbázis a tesztadatbazis.dump fájlban található. Ezt a fájlt egy Neo4j gráfadatbázisba kell importálni. Neo4j Aura esetében az adatbázis létrehozásakor kapott felhasználónevet, jelszót és URI-t az alkalamzás server.js fájlában kell megadni. 

\bigskip

A borítóképek FireBase Storage-be kerülnek feltöltésre. Ehhez telepíteni kell a firebase csomagot. Ezt követően be kell jelentkezni, initializálni a projektet, majd felállítani a storage-t. Az AdminNewBook.js fájlban kell átírni a projectId-t az aktuális projektnek megfelelően. Mivel a borítóképek elérési útvonala függ a projekt névtől, a tesztadatbázisban található könyvek boríóképei nem fognak megjelenni, ha a FireBase Storage, amin a képek vannak, nem elérhető.
